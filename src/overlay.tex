% overlay.tex
\documentclass[border=0mm]{standalone}
\usepackage{tikz}
\usetikzlibrary{calc}

% allow these to be overridden by \def on the command line:
\providecommand{\pw}{95}     % logical page width
\providecommand{\ph}{148}    % logical page height
\providecommand{\m}{5}       % cut‐mark length
\providecommand{\sheetW}{297}% sheet width
\providecommand{\sheetH}{210}% sheet height

\begin{document}
\begin{tikzpicture}[x=1mm,y=-1mm,line width=0.3pt]
  % compute centering offsets
  \pgfmathsetmacro{\offsetX}{(\sheetW-2*\pw)/2}
  \pgfmathsetmacro{\offsetY}{-(\sheetH-\ph)/2}

  % set the PDF bbox to the full sheet
  \useasboundingbox (0,0) rectangle (\sheetW,\sheetH);

  % now center your marks in a shifted scope
  \begin{scope}[shift={(\offsetX mm,\offsetY mm)}]
    % positions of pages and fold
    \pgfmathsetmacro{\leftX}{0}
    \pgfmathsetmacro{\centerX}{\pw}
    \pgfmathsetmacro{\rightX}{2*\pw}
    \pgfmathsetmacro{\topY}{0}
    \pgfmathsetmacro{\bottomY}{\ph}

    % vertical marks
    \foreach \x in {\leftX,\centerX,\rightX} {
      \draw (\x,\topY)    -- (\x,-\m);
      \draw (\x,\bottomY) -- (\x,{\bottomY+\m});
    }

    % horizontal marks
    \foreach \y in {\topY,\bottomY} {
      \draw (\leftX,\y)   -- (\leftX-\m,\y);
      \draw (\centerX,\y) -- (\centerX-\m,\y);
      \draw (\centerX,\y) -- (\centerX+\m,\y);
      \draw (\rightX,\y)  -- (\rightX+\m,\y);
    }
  \end{scope}
\end{tikzpicture}
\end{document}
